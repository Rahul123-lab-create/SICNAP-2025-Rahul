\documentclass{article}
\usepackage[utf8]{inputenc}
\usepackage{amsmath}
\usepackage{graphicx}

\title{SICNAP 2025 Internship Report}
\author{Rahul Adhikari}
\date{\today}

\begin{document}

\maketitle

\section{Introduction}
This is a brief report of the tasks completed during the Summer Internship in Computational Nuclear Astrophysics (SICNAP) 2025.

\section{Part A: Bash and Linux}
Completed all shell tasks including script creation and file editing with Vim. Below are the key actions performed:

\begin{itemize}
  \item Created directories and managed files with \texttt{mkdir}, \texttt{ls}, and \texttt{rm}
  \item Used commands like \texttt{grep}, \texttt{chmod}, \texttt{alias}, \texttt{date}, and \texttt{tar}
  \item Wrote and executed \texttt{.sh} scripts to automate tasks
\end{itemize}

\section{Part B: Python Programming} 
Practiced data types, control flow, functions, and file handling using Python.

\section{Part C: Git and GitHub}
Worked with branches, pushed to remote repositories, used stash and branch protection.

\section{Part D: LaTeX}
This report itself is created using LaTeX. It includes text formatting, equations, tables, and images.

See Table~\ref{tab:tools} for a summary of tools used during the internship.  
Figure~\ref{fig:sicnap} shows the tools used in this internship.

\begin{table}[h!]
\centering
\caption{Comparison of Tools Used in SICNAP}
\label{tab:tools}
\begin{tabular}{|l|l|l|}
\hline
\textbf{Tool} & \textbf{Use Case} & \textbf{Environment} \\
\hline
Bash   & File handling, scripting     & Git Bash \\
Python & Programming logic            & VS Code \\
Git    & Version control              & GitHub \\
LaTeX  & Report writing               & Overleaf/VS Code \\
\hline
\end{tabular}
\end{table}

\begin{figure}[h!]
\centering
\includegraphics[width=0.6\textwidth]{sicnap.png}
\caption{Overview of Tools in SICNAP}
\label{fig:sicnap}
\end{figure}

\section*{Mathematical Equations}

This equation uses the \texttt{equation} environment:

\begin{equation}
E = mc^2
\label{eq:einstein}
\end{equation}

As shown in Equation~\ref{eq:einstein}, energy is proportional to mass.

This pair of equations uses the \texttt{align} environment:

\begin{align}
a^2 + b^2 &= c^2 \\
x &= \frac{-b \pm \sqrt{b^2 - 4ac}}{2a}
\label{eq:maths}
\end{align}

See Equation~\ref{eq:maths} for more examples.

\end{document}
